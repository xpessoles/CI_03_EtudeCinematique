\documentclass[10pt,oneside]{article}
\input{style/coursHeadings}


%Si le boolen xp est vrai : compilation pour xabi
%Sinon compilation Damien
\newboolean{xp}
\setboolean{xp}{true}

\newboolean{prof}
\setboolean{prof}{true}

\def\xxtitre{\ifthenelse{\boolean{xp}}{
CI 3 -- CIN : Étude du comportement cinématique des systèmes}{
}}

\def\xxsoustitre{\ifthenelse{\boolean{xp}}{
Chapitre 7 -- }{
}}


\def\xxauteur{\ifthenelse{\boolean{xp}}{
\noindent 2013 -- 2014 \\
Xavier \textsc{Pessoles}}{
}}


\def\xxpied{\ifthenelse{\boolean{xp}}{
CI 3 : CIN -- Cours \\
Ch 7 : Torseurs -- \ifthenelse{\boolean{prof}}{P}{E}%
}{
}}

\usepackage[%
    pdftitle={CIN : Cinématique du point},
    pdfauthor={Xavier Pessoles},
    colorlinks=true,
    linkcolor=blue,
    citecolor=magenta]{hyperref}


\usepackage{pifont}
\sloppy
\hyphenpenalty 10000


\begin{document}


\input{style/entete1}

\begin{center}
 \huge\textsc{\xxtitre}

\end{center}

\begin{center}
 \LARGE\textsc{\xxsoustitre}
\end{center}

\begin{savoir}
\textbf{Savoirs :}
\begin{itemize}
\item Mod-- C12 -- S2 : Identifier, dans le cas du contact ponctuel, le vecteur vitesse de glissement ainsi que les vecteurs rotation de roulement et de pivotement.
\end{itemize}
\end{savoir}

\begin{center}
\begin{tabular}{ccc}
%\includegraphics[height=3.5cm]{png/fig1} &
%\includegraphics[height=3.5cm]{png/fig2}\\
%\textit{Modèle CAO d'un} & \textit{Modélisation par}\\
%\textit{arbre à came} & \textit{schéma cinématique}\\
\end{tabular}
\end{center}


%\begin{savoir}
%\textsc{Savoirs :}
%\begin{itemize}
%\item Connaître les principes de la cinématique des contacts (adhérence ou glissement, roulement et pivotement)
%\end{itemize}
%\end{savoir}

\setlength{\parskip}{0ex plus 0.2ex minus 0ex}
 \renewcommand{\contentsname}{}
 \renewcommand{\baselinestretch}{1}

\tableofcontents

 \renewcommand{\baselinestretch}{1.2}
\setlength{\parskip}{2ex plus 0.5ex minus 0.2ex}

% \vspace{1cm}
\textit{Ce document est en évolution permanente. Merci de signaler toutes
erreurs ou coquilles.}


\section{Définitions}
\subsection{Torseurs}

\begin{defi}
Un torseur est constitué :
\begin{itemize}
\item d'un vecteur résultant $\vect{R}$;
\item d'un champ de vecteur *** $\mathcal{M]$ tel que pour tout couples de points $(A,B)$ on a : 
$$
\end{itemize}
\end{defi}


\begin{thebibliography}{2}
\bibitem[1]{gdi} André Chevalier, Guide du dessinateur Industriel, Éditions Hachette Technique.
\end{thebibliography}
\end{document}


\end{document}


