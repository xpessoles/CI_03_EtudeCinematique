\documentclass[10pt,oneside]{article}
\input{style/coursHeadings}


%Si le boolen xp est vrai : compilation pour xabi
%Sinon compilation Damien
\newboolean{xp}
\setboolean{xp}{true}

\newboolean{prof}
\setboolean{prof}{true}

\def\xxtitre{\ifthenelse{\boolean{xp}}{
CI 3 -- CIN : Étude du comportement cinématique des systèmes}{
}}

\def\xxsoustitre{\ifthenelse{\boolean{xp}}{
Chapitre 7 -- }{
}}


\def\xxauteur{\ifthenelse{\boolean{xp}}{
\noindent 2013 -- 2014 \\
Xavier \textsc{Pessoles}}{
}}


\def\xxpied{\ifthenelse{\boolean{xp}}{
CI 3 : CIN -- Cours \\
Ch 7 : Torseurs -- \ifthenelse{\boolean{prof}}{P}{E}%
}{
}}

\usepackage[%
    pdftitle={CIN : Cinématique du point},
    pdfauthor={Xavier Pessoles},
    colorlinks=true,
    linkcolor=blue,
    citecolor=magenta]{hyperref}


\usepackage{pifont}
\sloppy
\hyphenpenalty 10000


\begin{document}


\input{style/entete1}

\begin{center}
 \huge\textsc{\xxtitre}

\end{center}

\begin{center}
 \LARGE\textsc{\xxsoustitre}
\end{center}

\begin{savoir}
\textbf{Savoirs :}
\begin{itemize}
\item Mod-- C12 -- S2 : Identifier, dans le cas du contact ponctuel, le vecteur vitesse de glissement ainsi que les vecteurs rotation de roulement et de pivotement.
\end{itemize}
\end{savoir}

\begin{center}
\begin{tabular}{ccc}
%\includegraphics[height=3.5cm]{png/fig1} &
%\includegraphics[height=3.5cm]{png/fig2}\\
%\textit{Modèle CAO d'un} & \textit{Modélisation par}\\
%\textit{arbre à came} & \textit{schéma cinématique}\\
\end{tabular}
\end{center}


%\begin{savoir}
%\textsc{Savoirs :}
%\begin{itemize}
%\item Connaître les principes de la cinématique des contacts (adhérence ou glissement, roulement et pivotement)
%\end{itemize}
%\end{savoir}

\setlength{\parskip}{0ex plus 0.2ex minus 0ex}
 \renewcommand{\contentsname}{}
 \renewcommand{\baselinestretch}{1}

\tableofcontents

 \renewcommand{\baselinestretch}{1.2}
\setlength{\parskip}{2ex plus 0.5ex minus 0.2ex}

% \vspace{1cm}
\textit{Ce document est en évolution permanente. Merci de signaler toutes
erreurs ou coquilles.}


\section{Définitions}
\subsection{Torseurs}

\begin{defi}
Un torseur est constitué :
\begin{itemize}
\item d'un vecteur résultant $\vect{R}$;
\item d'un champ de vecteur *** $\mathcal{M}$ tel que pour tout couples de points $(A,B)$ on a : 
$$*** $$
\end{itemize}
\end{defi}

Cette relation est appelée relation caractéristique des torseurs ou relation de Varignon.

Réciproquement, tout champ de vecteurs   qui vérifie la relation de Varignon pour tout couple de points $A$ et $B$, le vecteur $\vect{R}$ étant indépendant du bipoint $(A,B)$, est le champ de vecteurs d'un torseur.

\subsection{Expression en un point}
Si le champ de vecteurs est connu en un point, on peut le connaître en tout autre point en utilisant la relation de Varignon. Le torseur sera donc noté :
$$
\torseur{\mathcal{T}}=\torseurcol{}{}{}{}{}{}{}
\quad \text{au point }A, \quad 
\torseur{\mathcal{T}}=\torseurcol{}{}{}{}{}{}{}
\quad \text{au point }B \text{ avec } ***
$$
On appelle :
\begin{itemize}
\item $\vect{R}$ la résultante du torseur;
\item $\vect{\mathcal{M}_A}$ le moment au point $A$ du torseur. Ce nom est donné par analogie avec la définition du moment en un point d'un pointeur;
\item $\vect{R}$ et $\vect{\mathcal{M}_A}$ sont les éléments de réduction du torseur. Ce sont deux vecteurs dont on peut faire une représentation graphique au point $A$.
\end{itemize}

\subsection{Notation des torseurs}
Les torseurs sont utilisés en cinématique, statique, dynamique, métrologie, \textit{etc}. Le torseur est exprimé en général par une lettre majuscule significative de ce que l'on étudie. Cette lettre est encadrée d'accolades :
$$
\torseur{\mathcal{T}}, \torseur{\mathcal{V}(2/1)}, \torseur{\mathcal{F}(1\rightarrow 2)}, \torseur{\mathcal{U}(1/2)}, \torseur{\mathcal{C}(1/2)}, \torseur{\mathcal{D}(1/2)}
$$

Pour travailler sur ces torseurs, il est plus commode de le faire avec ses éléments de réduction. Le moment dépend du point d'expression. Il faut donc que la notation précise bien ce point :

$$
\torseur{\mathcal{T}} = \torseurl{\vect{R}}{\vect{\mathcal{M}_A}}
\quad \text{ou}\quad
\torseur{\mathcal{T}} =\torseur{\vect{R},\vect{\mathcal{M}_A}} 
$$

$$
\torseur{\mathcal{T}} = \torseurl{\vect{R}}{3\vect{x}+2\vect{y}}{A}
\quad \text{ou}\quad
\torseur{\mathcal{T}} =\torseur{\vect{R},3\vect{x}+2\vect{y}}_A
$$

Le point $A$ apparaît à l'intérieur ou à l'extérieur des accolades (par exemple, en indice à gauche ou à droite des accolades).

On peut aussi exprimer des éléments de réduction en projection sur une même base $\mathcal{B}$ et les écrire en colonne en précisant bien la base de projection :

$$
\torseur{\mathcal{T}}=\torseurcol{-6}{a}{1}{3}{2}{0}{A,\mathcal{B}}
$$
\begin{thebibliography}{2}
\bibitem[1]{gdi} André Chevalier, Guide du dessinateur Industriel, Éditions Hachette Technique.
\end{thebibliography}
\end{document}


\end{document}


