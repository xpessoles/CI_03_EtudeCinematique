\documentclass[10pt,oneside]{article}
\input{coursHeadings}

\usepackage[%
    pdftitle={CIN -- Torseurs -- Travaux Dirigés},
    pdfauthor={Xavier Pessoles},
    colorlinks=true,
    linkcolor=blue,
    citecolor=magenta]{hyperref}



% \makeatletter \let\ps@plain\ps@empty \makeatother
%% DEBUT DU DOCUMENT
%% =================
\sloppy
\hyphenpenalty 10000

\newcommand{\Pointilles}[1][3]{%
\multido{}{#1}{\makebox[\linewidth]{\dotfill}\\[\parskip]
}}


\begin{document}


\newboolean{prof}
\setboolean{prof}{false}
%------------- En tetes et Pieds de Pages ------------
\pagestyle{fancy}
\renewcommand{\headrulewidth}{0pt}

\fancyhead{}
\fancyhead[L]{%
\noindent\noindent\begin{minipage}[c]{2.6cm}
%Lycée Rouvière PTSI
\includegraphics[width=2cm]{png/logo_ptsi.png}%
\end{minipage}
}

\fancyhead[C]{\rule{12cm}{.5pt}}

\fancyhead[R]{%
\begin{minipage}[c]{3cm}
\begin{flushright}
\footnotesize{\textit{\textsf{Sciences Industrielles\\ de l'Ingénieur}}}%
\end{flushright}
\end{minipage}
}

\renewcommand{\footrulewidth}{0.2pt}

\fancyfoot[C]{\footnotesize{\bfseries \thepage}}
\fancyfoot[L]{\footnotesize{2013 -- 2014} \\ X. \textsc{Pessoles}}
\ifthenelse{\boolean{prof}}{%
\fancyfoot[R]{\footnotesize{CI 3 : CIN -- TD} \\ \footnotesize{Ch 7 : Torseurs -- P}}
}{%
\fancyfoot[R]{\footnotesize{CI 3 : CIN -- TD} \\ \footnotesize{Ch 7 : Torseurs -- E}}
}


%\begin{center}
%\textit{Centre d'intérêt}
%\end{center}



\begin{center}
 \Large\textsc{CI 3 -- CIN : Étude du comportement cinématique des systèmes}
\end{center}

\begin{center}
 \large\textsc{Chapitre 7 -- Torseurs}
\end{center}

\begin{center}
\textsc{Travaux Dirigés} 
\end{center}

\vspace{.5cm}

\subsection*{Exercice 1 -- Came plate}
\setcounter{subparagraph}{0}
\begin{flushright}
\textit{D'après ressources de Jean-Pierre Pupier.} 
\end{flushright}
La came plate 1 tourne et transmet un mouvement de translation alternatif à la pièce 2. Il y a une liaison sphère -- plan de centre A.

\begin{itemize}
\item On associe au bâti 0 un repère $\mathcal{R}_0=\left(H,\vect{x_0},\vect{y_0},\vect{z_0}, \right)$.
\item On associe à la came plate 1 un repère  $\mathcal{R}_1=\left(H,\vect{x_0},\vect{y_1},\vect{z_1} \right)$. 
\item On pose  $\alpha=\left( \vect{y_0}, \vect{y_1}\right)$.
\item Un repère $\mathcal{R}'_1$ est également lié à 1 : $\mathcal{R}'_1=\left(H,\vect{x'_1},\vect{y'_1},\vect{z_1} \right)$. 
\item Ce repère est tel que $\vect{x'_1}$ est perpendiculaire à la surface plane de 1 sur laquelle appuie la sphère.
\item On pose $\theta= \left( \vect{x_0}, \vect{x'_1}\right)$. $\theta$ est constant.
\item On associe à la pièce 2 un repère  $\mathcal{R}_2=\left(A,\vect{x_0},\vect{y_0},\vect{z_0} \right)$. 
\item On pose $\vect{HA}=R\vect{y_0}+\lambda\vect{x_0}$ où $R$ est constant et $\lambda$ variable.
\end{itemize}

\begin{center}
\includegraphics[width=.6\textwidth]{png/fig_01}
\end{center}

\subparagraph{}
\textit{Exprimer les torseurs $\torseur{\mathcal{V}(1/0)}$ et $\torseur{\mathcal{V}(2/0)}$. Préciser les noms de ces deux types de torseurs.}
\ifthenelse{\boolean{prof}}{
\begin{corrige}
\end{corrige}
}{}

\subparagraph{}
\textit{Exprimer le torseur $\torseur{\mathcal{V}(2/1)}$.}
\ifthenelse{\boolean{prof}}{
\begin{corrige}
\end{corrige}
}{}


\subparagraph{}
\textit{Trouver la relation entre $\dot{\lambda}$ et $\dot{\alpha}$.}
\ifthenelse{\boolean{prof}}{
\begin{corrige}
\end{corrige}
}{}


\subparagraph{}
\textit{Trouver l'expression de $\lambda$ et exprimer la course de 2 par rapport à 0. }
\ifthenelse{\boolean{prof}}{
\begin{corrige}
\end{corrige}
}{}

\subparagraph{}
\textit{Trouver la vitesse de glissement $\vectv{A}{2}{1}$ dans $\mathcal{B}'_1$ sans faire intervenir $\lambda$ (et sa dérivée).}
\ifthenelse{\boolean{prof}}{
\begin{corrige}
\end{corrige}
}{}


\end{document}
\setcounter{subparagraph}{0}

\subparagraph{}
\textit{}
\ifthenelse{\boolean{prof}}{
\begin{corrige}
\end{corrige}
}{}
