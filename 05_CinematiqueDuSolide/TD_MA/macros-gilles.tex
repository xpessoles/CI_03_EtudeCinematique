%%UTF8
%% macros-mecanique version 26-mai-2013
%\usepackage[francais, frenchb]{babel}
%\usepackage[T1]{fontenc}
%\usepackage[utf8]{inputenc}
%\usepackage{url}
%\usepackage{multicol}
%\usepackage{graphicx}
%\usepackage{xargs}
%%\usepackage[usenames,svgnames,dvipsnames]{xcolor}  %pas pour AMC
%\usepackage[leqno, fleqn]{amsmath}
%\usepackage{amssymb}
%\usepackage{mathtools}
%\usepackage{wrapfig}
%\usepackage{enumerate}
%\usepackage{longtable}
%\usepackage{tabularx}
\usepackage{tikz}
%\usetikzlibrary{shapes}
%\usetikzlibrary{calc}
%\usepackage{enumitem}
%\usepackage{fancyhdr}
%\usepackage{caption}
%\usepackage{subcaption}
%\usepackage{ifthen}
%\usepackage{calc}
%\usepackage{fp}
%\usepackage{multido}
%\usepackage{lscape}
%\usepackage{titling}
%\usepackage{cancel}
%\usepackage{geometry}
%\usepackage{float}
%%\usepackage{palatino}  %pas pour AMC
%\usepackage{hyperref}
%\usepackage{subscript}
%\usepackage{movie15}
%\usepackage{siunitx}  %pas pour AMC
%\usepackage{auto-greek}
%\usepackage{floatfig}




% Xavier
%\usepackage[adobe-utopia]{mathdesign}
%\usepackage{framed}
%\usepackage[normalem]{ulem}
%\usepackage{titlesec}
%\usepackage{vmargin}
%\usepackage{longtable}
%\usepackage{subfig}
%\usepackage{color}
%\usepackage{colortbl}
%
%
\definecolor{gris25}{gray}{0.75}
\definecolor{gris15}{gray}{0.85}
%\definecolor{bleu}{RGB}{18,33,98}
%\definecolor{bleuf}{RGB}{42,94,171}
%\definecolor{bleuc}{RGB}{231,239,247}
%\definecolor{rougef}{RGB}{185,18,27}
%\definecolor{rougec}{RGB}{255,230,231}
%\definecolor{vertf}{RGB}{103,126,82}
%\definecolor{vertc}{RGB}{220,255,191}
%

%Symbole degres \deg
\renewcommand{\deg}{\ensuremath{^\circ}\xspace}

%norme d'un vecteur
\newcommand{\norme}[1]{\ensuremath{\|#1 \|}}
%grande fleche sur les vecteurs : \gv
\newcommand{\gv}[1]{\ensuremath{\overrightarrow{#1}}}
%vecteurs de base raccourcis
\newcommand{\vx}[1]{\ensuremath{\vec{x}_{#1}}}
\newcommand{\vy}[1]{\ensuremath{\vec{y}_{#1}}}
\newcommand{\vz}[1]{\ensuremath{\vec{z}_{#1}}}

%coordonnees de vecteurs		  			
\newcommand{\coordvecteur}[3]{\begin{pmatrix} #1 \\ #2\\ #3 \end{pmatrix}}   
% pour les torseurs
%indice en bas a gauche (torseurs) \indicegauche
\newcommand{\indicegauche}[2]{{\vphantom{#2}}_{#1}#2}
% nom d'un torseur d'action meca \nomtorseur
\newcommand{\nomtorseur}[2]{\left\{ \mathcal{#1}_{#2} \right\}}

%Torseur d'action mécanique
\newcommand{\torseuram}[2]{\ensuremath{\left\{ \mathcal{T}_{#1 \to #2} \right\}}}


% torseur sous la forme développée en coordonnees pluckeriennes
\newcommand{\coordtorseur}[7]{\ensuremath{\indicegauche{#1}{\begin{Bmatrix} #2 & #5 \\ #3 & #6 \\ #4 & #7 \end{Bmatrix}}}}
% torseur phantom
\newcommand{\torseurphantom}{\ensuremath{\vphantom{\indicegauche{0}{\begin{Bmatrix} 0 & 0 \\ 0 & 0 \\ 0 & 0 \end{Bmatrix}}}}}
\newcommand{\composanteOmegaTorseur}[3]{\omega_{#1_{#2#3}}}
\newcommand{\composanteVitesseTorseur}[3]{v_{#1_{#2#3}}}
% torseur sous la forme developpee en coordonnees pluckeriennes avec base xyz
\newcommand{\coordtorseurxyz}[7]{\ensuremath{\indicegauche{#1}{\begin{Bmatrix} #2 & #5 \\ #3 & #6 \\ #4 & #7 \end{Bmatrix}_{(\vec{x},\vec{y},\vec{z})}}}}
\newcommand{\torcindev}[9]{
\coordtorseur{#1}{\ifthenelse{#2=1}{\composanteOmegaTorseur{x}{#8}{#9}}{0}}{\ifthenelse{#3=1}{\composanteOmegaTorseur{y}{#8}{#9}}{0}}{\ifthenelse{#4=1}{\composanteOmegaTorseur{z}{#8}{#9}}{0}}{\ifthenelse{#5=1}{\composanteVitesseTorseur{x}{#8}{#9}}{0}}{\ifthenelse{#6=1}{\composanteVitesseTorseur{y}{#8}{#9}}{0}}{\ifthenelse{#7=1}{\composanteVitesseTorseur{z}{#8}{#9}}{0}}
}
% d droit pour la differentielle
\newcommand{\dd}{\mathrm{d}}
%vecteur vitesse
\newcommand{\vecvit}[3]{\ensuremath{\gv{V}_{#1\in #2 / #3}}}
%vecteur taux de rotation
\newcommand{\vecrot}[2]{\ensuremath{\gv{\Omega}_{#1 / #2}}}
%vecteur acceleration
\newcommand{\vecacc}[3]{\ensuremath{\gv{\Gamma}_{#1 \in #2 / #3}}}
% vecteur force
\newcommand{\vecforce}[3]{\ensuremath{\gv{#1}_{#2 \to #3}}}
% vecteur moment
\newcommand{\vecmoment}[2]{\ensuremath{\gv{M}_{#1(#2)}}}
% Mouvement
\newcommand{\mvt}[2]{\ensuremath{\text{Mvt}_{#1 / #2}}}
% Torseur cinematique
\newcommand{\torcin}[2]{\ensuremath{\left\{ \mathcal{V}_{#1 / #2} \right\}}}
% derivee d'un vecteur par rapport au temps, et dans un referentiel donne
\newcommand{\derivect}[2]{\ensuremath{\left( \frac{\dd #1}{\dd t} \right)_{#2}}}
% dérivée d'un vecteur par rapport au temps, et dans un référentiel donné, grande fraction (texte)
\newcommand{\gderivect}[2]{\ensuremath{\left( \dfrac{\dd #1}{\dd t} \right)_{#2}}}
% forme littérale d'un torseur
\newcommand{\torlit}[3]{{\indicegauche{#1}{\begin{Bmatrix} #2 \\ #3  \end{Bmatrix}}}}
% Trajectoire d'un point
\newcommand{\traj}[3]{\ensuremath{T_{#1 \in #2/#3}}}
% derivee scalaire par rapport au temps
\newcommand{\derit}[1]{\frac{\mathrm{d}#1}{\mathrm{d}t}}
% différentielle droite \ud
\newcommand{\ud}{\mathrm{d}}
% calcul d'un produit vectoriel par les composantes (entiers relatifs uniquement)
\newcommand{\prodvec}[6]{\ensuremath{
\FPeval\respvx{trunc(((#2*#6)-(#3*#5)),0)}
\FPeval\respvy{trunc(((#3*#4)-(#1*#6)),0)}
\FPeval\respvz{trunc(((#1*#5)-(#2*#4)),0)}
\begin{pmatrix}
 \respvx \\
 \respvy \\
 \respvz 
\end{pmatrix}
}}
% systeme d'equations
\newcommand{\systeq}[1]{\ensuremath{\left\{ \begin{aligned} #1 \end{aligned}\right.}}

%pour générer des copies à trous
\newcommand{\acompleter}[1]{\ifthenelse{\boolean{version-a-trous}}{\phantom{{\Large{#1\xspace}}}}{\textcolor{red}{#1}}}

% mise en forme
% exemple
%\newcommand{\exemple}[1]{\paragraph{Exemple :} #1}
% exemple
\newcommand{\exemples}[1]{\paragraph{Exemples :} #1}


%%%%%%% mise en forme de Xavier
\newenvironment{definition}[1][\hsize]%
{%
    \def\FrameCommand%
    {%
\rotatebox{90}{\textit{\textsf{Definition\\}}} 
        {\color{bleuf}\vrule width 3pt}%
        \hspace{0pt}%must no space.
        \fboxsep=\FrameSep\colorbox{bleuc}%
    }%
    \MakeFramed{\hsize#1\advance\hsize-\width\FrameRestore}%
}%
{\endMakeFramed}%
%%%%%%%%%%%%%%%%%%%


% pour encadrer dans un alignement
\makeatletter
\newlength{\boxed@align@width}
\newcommand{\boxedalign}[2]{
#1 & \setlength{\boxed@align@width}{\widthof{$\displaystyle#1$}+1.4pt+1.4pt}
\hspace{-\boxed@align@width}\boxed{\vphantom{#1}\hspace{\boxed@align@width}#2}}
\makeatother
%%%%

%%%%%%%%%%%%%%%%%%%%%%%%%%%%%%%%%

%Commande pour tracer des figures planes: \figureplane
\newcommand\figureplane[9]{
\begin{tikzpicture}[scale=#9]
\draw[thick,->,#7] (0,0) -- (0:1) node[anchor=north west]{$\vec{#1}$};
\draw[thick,->,#7] (0,0) -- (90:1) node[anchor=north east]{$\vec{#3}$}; 
\draw[thick,->,#8] (0,0) -- (20:1) node[anchor=south east]{$\vec{#2}$}; 
\draw[thick,->,#8] (0,0) -- (110:1) node[anchor=north east]{$\vec{#4}$}; 
\fill[white] (0,0) circle (0.05);
\draw[thick,#7] (0,0) circle (0.05);
\fill[#7] (0,0) circle (0.01);
\draw (-0.05,0) node[anchor=north east] {$\vec{#5}$};
\draw[thick,->] (0.4,0) arc (0:20:0.4);
\draw[thick,->] (0,0.4) arc (90:110:0.4);
\draw (0.4,0.08) node[anchor=west] {$#6$};
\draw (-0.08,0.4) node[anchor=south] {$#6$};
\end{tikzpicture}
}

%couleurs des figures planes pour ne pas dépasser neuf arguments
\newcommand\couleursfigureplane[2]{%
    \def\premierecouleur{#1}%
    \def\deuxiemecouleur{#2}%
}


%Commande pour tracer des figures planes avec reperes: \figureplanerep
\newcommand\figureplanerep[9]{
\begin{tikzpicture}[scale=#9]
\draw[thick,->,\premierecouleur ] (0,0) -- (0:1) node[anchor=north west]{$\vec{#1}$};
\draw[thick,->,\premierecouleur ] (0,0) -- (90:1) node[anchor=north east]{$\vec{#3}$}; 
\draw[thick,->,\deuxiemecouleur ] (0,0) -- (20:1) node[anchor=south east]{$\vec{#2}$}; 
\draw[thick,->,\deuxiemecouleur ] (0,0) -- (110:1) node[anchor=north east]{$\vec{#4}$}; 
\fill[white] (0,0) circle (0.05);
\draw[thick,\premierecouleur] (0,0) circle (0.05);
\fill[\premierecouleur] (0,0) circle (0.01);
\draw[\premierecouleur] (-0.05,0) node[anchor=north east] {$\vec{#5}$};
\draw[thick,->] (0.4,0) arc (0:20:0.4);
\draw[thick,->] (0,0.4) arc (90:110:0.4);
\draw[thick,\premierecouleur] (0:0.5) arc (180:360:0.15) ;
\draw[\premierecouleur] (0.65,-0.07) node{\ensuremath{#7}};
\draw[thick,\deuxiemecouleur] (20:0.8) arc (20:200:0.15);
\draw[\deuxiemecouleur] (0.59,0.29) node{\ensuremath{#8}};
\draw (0.4,0.08) node[anchor=west] {$#6$};
\draw (-0.08,0.4) node[anchor=south] {$#6$};
\end{tikzpicture}
}

%Commande pour tracer des figures planes autour de z avec reperes: \figureplanerepz
\newcommand\figureplanerepz[9]{
\begin{tikzpicture}[scale=#9]
\draw[thick,->, #7] (0,0) -- (0:1) node[anchor=north west]{$\vec{x}_{#1}$};
\draw[thick,->, #7] (0,0) -- (90:1) node[anchor=north east]{$\vec{y}_{#1}$}; 
\draw[thick,->, #8] (0,0) -- (20:1) node[anchor=south east]{$\vec{x}_{#2}$}; 
\draw[thick,->, #8] (0,0) -- (110:1) node[anchor=north east]{$\vec{y}_{#2}$}; 
\fill[white] (0,0) circle (0.05);
\draw[thick, #7] (0,0) circle (0.05);
\fill[#7] (0,0) circle (0.01);
\draw[#7] (-0.05,0) node[anchor=north east] {$\vec{z}_{#3}$};
\draw[thick,->] (0.4,0) arc (0:20:0.4);
\draw[thick,->] (0,0.4) arc (90:110:0.4);
\draw (0.4,0.08) node[anchor=west] {$#4$};
\draw (-0.08,0.4) node[anchor=south] {$#4$};
\draw[#7] [thick] (0:0.5) arc (180:360:0.15) ;
\draw[#7] (0.65,-0.07) node{\ensuremath{#5}};
\draw[#8] [thick] (20:0.8) arc (20:200:0.15);
\draw[#8] (0.59,0.29) node{\ensuremath{#6}};
\end{tikzpicture}
}

%Commande pour tracer des figures planes autour de x avec reperes: \figureplanerepx
\newcommand\figureplanerepx[9]{
\begin{tikzpicture}[scale=#9]
\draw[thick,->, #7] (0,0) -- (0:1) node[anchor=north west]{$\vec{y}_{#1}$};
\draw[thick,->, #7] (0,0) -- (90:1) node[anchor=north east]{$\vec{z}_{#1}$}; 
\draw[thick,->, #8] (0,0) -- (20:1) node[anchor=south east]{$\vec{y}_{#2}$}; 
\draw[thick,->, #8] (0,0) -- (110:1) node[anchor=north east]{$\vec{z}_{#2}$}; 
\fill[white] (0,0) circle (0.05);
\draw[thick, #7] (0,0) circle (0.05);
\fill[#7] (0,0) circle (0.01);
\draw[#7] (-0.05,0) node[anchor=north east] {$\vec{x}_{#3}$};
\draw[thick,->] (0.4,0) arc (0:20:0.4);
\draw[thick,->] (0,0.4) arc (90:110:0.4);
\draw (0.4,0.08) node[anchor=west] {$#4$};
\draw (-0.08,0.4) node[anchor=south] {$#4$};
\draw[#7] [thick] (0:0.5) arc (180:360:0.15) ;
\draw[#7] (0.65,-0.07) node{\ensuremath{#5}};
\draw[#8] [thick] (20:0.8) arc (20:200:0.15);
\draw[#8] (0.59,0.29) node{\ensuremath{#6}};
\end{tikzpicture}
}

%Commande pour tracer des figures planes autour de y avec reperes: \figureplanerepy
\newcommand\figureplanerepy[9]{
\begin{tikzpicture}[scale=#9]
\draw[thick,->, #7] (0,0) -- (0:1) node[anchor=north west]{$\vec{z}_{#1}$};
\draw[thick,->, #7] (0,0) -- (90:1) node[anchor=north east]{$\vec{x}_{#1}$}; 
\draw[thick,->, #8] (0,0) -- (20:1) node[anchor=south east]{$\vec{z}_{#2}$}; 
\draw[thick,->, #8] (0,0) -- (110:1) node[anchor=north east]{$\vec{x}_{#2}$}; 
\fill[white] (0,0) circle (0.05);
\draw[thick, #7] (0,0) circle (0.05);
\fill[#7] (0,0) circle (0.01);
\draw[#7] (-0.05,0) node[anchor=north east] {$\vec{y}_{#3}$};
\draw[thick,->] (0.4,0) arc (0:20:0.4);
\draw[thick,->] (0,0.4) arc (90:110:0.4);
\draw (0.4,0.08) node[anchor=west] {$#4$};
\draw (-0.08,0.4) node[anchor=south] {$#4$};
\draw[#7] [thick] (0:0.5) arc (180:360:0.15) ;
\draw[#7] (0.65,-0.07) node{\ensuremath{#5}};
\draw[#8] [thick] (20:0.8) arc (20:200:0.15);
\draw[#8] (0.59,0.29) node{\ensuremath{#6}};
\end{tikzpicture}
}

%%% Graphe des liaisons%%%%%%%%%%%%

%\begin{tikzpicture}[scale=1]
%\tikzstyle{sommet}=[draw,ellipse,minimum height=1.5cm,fill=gray!20]
%\tikzstyle{liaison}=[text width=2cm, text centered]
%\node[sommet] (0) at (360/3*0:3) {Bâti 0};
%\node[sommet] (1) at (360/3*1:3) {Coulisseau 1};
%\node[sommet] (2) at (360/3*2:3) {Panneau 2};
%\draw (1) to[bend right=20] (2);
%\draw (2) to[bend right=20] (0);
%\draw (0) to[bend right=20] (1);
%\node[liaison] at (360/3*0.4:3.2) {Glissière de direction \vx{0}} ;
%\node[liaison] at (360/3*1.5:3.2) {Ponctuelle de normale $(B,\vx{2})$} ;
%\node[liaison] at (360/3*2.6:3.3) {Pivot d'axe $(C,\vz{0})$} ;
%\end{tikzpicture}

%Commande pour tracer des bases isometriques: \baseiso
\newcommand\baseiso[5]{
\begin{tikzpicture}[scale=#5*0.82]
\draw[->] (0,0) -- (-30:1) node[anchor=north west]{$\vec{#1}$};
\draw[->] (0,0) -- (90:1) node[anchor=east]{$\vec{#2}$}; 
\draw[->] (0,0) -- (210:1) node[anchor=north east]{$\vec{#3}$};
\draw node[anchor=south east]{#4};
\end{tikzpicture}
}


%Commande pour tracer des bases planes: \baseplane
\newcommand\baseplane[4]{
\begin{tikzpicture}[scale=#4]
\draw[->] (0,0) -- (0:1) node[anchor=north west]{$\vec{#1}$};
\draw[->] (0,0) -- (90:1) node[anchor=south east]{$\vec{#2}$}; 
\draw node[anchor=north east]{#3};
\end{tikzpicture}
}

%Commande pour tracer des bases en 2D avec vecteur sortant, tournées d'un certain angle : \basevecteursortant
\newcommand\basevecteursortant[5]{
\begin{tikzpicture}[scale=#5]
\draw[->] (0,0) -- (0+#4:1) node[anchor=north west]{$\vec{#1}$};
\draw[->] (0,0) -- (90+#4:1) node[anchor=south east]{$\vec{#2}$}; 
\fill[white] (0,0) circle (0.05);
\draw[thick] (0,0) circle (0.05);
\fill (0,0) circle (0.01);
\draw (-0.05,0) node[anchor=north east] {$\vec{#3}$};
\end{tikzpicture}
}

%Commande pour tracer des bases en 2D avec vecteur rentrant, tournées d'un certain angle : \basevecteurrentrant
\newcommand\basevecteurrentrant[5]{
\begin{tikzpicture}[scale=#5]
\draw[->] (0,0) -- (#4:1) node[anchor=north west]{$\vec{#1}$};
\draw[->] (0,0) -- (90+#4:1) node[anchor=south east]{$\vec{#2}$}; 
\fill[white] (0,0) circle (0.05);
\draw[thick] (0,0) circle (0.05);
\draw (225:0.05) -- (45:0.05);
\draw (135:0.05) -- (-45:0.05);
\draw (-0.05,0) node[anchor=north east] {$\vec{#3}$};
\end{tikzpicture}
}

%Commande pour tracer des bases planes tournées d'un certain angle: \baseplanerot
\newcommand\baseplanerot[5]{
\begin{tikzpicture}[scale=#5]
\draw[->] (0,0) -- (0+#4:1) node[anchor=north west]{$\vec{#1}$};
\draw[->] (0,0) -- (90+#4:1) node[anchor=south east]{$\vec{#2}$}; 
\draw node[anchor=north east]{#3};
\end{tikzpicture}
}

%Commande pour tracer des tableaux de ddl : tabddl
\newcommand\tabddl[6]{
\begin{tabular}{|c|c|}
\hline 
\ensuremath{#1} & \ensuremath{#4} \\ 
\hline 
\ensuremath{#2} & \ensuremath{#5} \\ 
\hline 
\ensuremath{#3} & \ensuremath{#6} \\ 
\hline 
\end{tabular} 
}

%Commande pour tracer des tableaux de ddl vides : tabddlvide
\newcommand\tabddlvide{
\begin{tabular}{|c|c|}
\hline 
$\phantom{T_x}$ & $\phantom{T_x}$ \\ 
\hline 
$\phantom{T_x}$ & $\phantom{T_x}$ \\ 
\hline 
$\phantom{T_x}$ & $\phantom{T_x}$ \\ 
\hline 
\end{tabular} 
}


%%% figure pour projeter un vecteur avec un angle quelconque \figureprojection{x}{y}{z}{u}{v}{\theta}{40}{4}
\newcommand\figureprojection[8]{
\begin{tikzpicture}[scale=#8]
\draw[thick,->] (0,0) -- (0:1); 
\draw (0:1.1) node {$\vec{#1}$};
\draw[thick,->] (0,0) -- (90:1);
\draw (90:1.1) node {$\vec{#2}$}; 
\draw[thick,->] (0,0) -- (#7:1);
\draw (#7:1.1) node {$\vec{#4}$}; 
\draw[thick,->] (0,0) -- (90+#7:1);
\draw (90+#7:1.1) node {$\vec{#5}$}; 
\fill[white] (0,0) circle (0.05);
\draw[thick] (0,0) circle (0.05);
\fill (0,0) circle (0.01);
\draw (-0.05,0) node[anchor=north east] {$\vec{#3}$};
\draw[thick,->] (0.4,0) arc (0:#7:0.4);
\draw[thick,->] (0,0.4) arc (90:90+#7:0.4);
\draw (#7/2:0.5) node {$#6$};
\draw (#7/2+90:0.5) node {$#6$};
\end{tikzpicture}
}

\newcommand\laplace[1]{L(#1)}

\newcommand\baseisoxyz{\baseiso{x}{y}{z}{A}{2}}
\newcommand\baseplanexy{\baseplane{x}{y}{A}{2}}
\newcommand\baseplaneyz{\baseplane{y}{z}{A}{2}}
\newcommand\baseplanezx{\baseplane{z}{x}{A}{2}}