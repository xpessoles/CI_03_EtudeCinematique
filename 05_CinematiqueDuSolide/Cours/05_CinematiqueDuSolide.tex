\documentclass[10pt,oneside]{article}
\input{style/coursHeadings}


%Si le boolen xp est vrai : compilation pour xabi
%Sinon compilation Damien
\newboolean{xp}
\setboolean{xp}{true}

\newboolean{prof}
\setboolean{prof}{true}

\def\xxtitre{\ifthenelse{\boolean{xp}}{
CI 3 -- CIN : Étude du comportement cinématique des systèmes}{
}}

\def\xxsoustitre{\ifthenelse{\boolean{xp}}{
Chapitre 5 -- Cinématique du solide indéformable}{
}}


\def\xxauteur{\ifthenelse{\boolean{xp}}{
\noindent 2013 -- 2014 \\
Xavier \textsc{Pessoles}}{
}}


\def\xxpied{\ifthenelse{\boolean{xp}}{
CI 3 : CIN -- Cours \\
Ch 4 : Cinématique du solide -- \ifthenelse{\boolean{prof}}{P}{E}%
}{
}}

\usepackage[%
    pdftitle={CIN : Cinématique du solide},
    pdfauthor={Xavier Pessoles},
    colorlinks=true,
    linkcolor=blue,
    citecolor=magenta]{hyperref}


\usepackage{pifont}
\sloppy
\hyphenpenalty 10000


\begin{document}


\input{style/entete1}

\begin{center}
 \huge\textsc{\xxtitre}

\end{center}

\begin{center}
 \LARGE\textsc{\xxsoustitre}
\end{center}


\begin{minipage}[c]{.3\linewidth}
\begin{center}
%\includegraphics[height=2.5cm]{png/avion}

%\textit{Trainer Solo Sport \cite{cite1}} 
\end{center}
\end{minipage} \hfill
\begin{minipage}[c]{.3\linewidth}
\begin{center}
%\includegraphics[height=3.5cm]{png/moteur_3d}

%\textit{Modèle CAO d'un moteur de modélisme \cite{cite2}}
\end{center}
\end{minipage} \hfill
\begin{minipage}[c]{.3\linewidth}
\begin{center}
%\includegraphics[height=3.5cm]{png/moteur_3d_sch}

%\textit{Modélisation par schéma cinématique}
\end{center}
\end{minipage}


\begin{savoir}
\textbf{Savoirs :}
\begin{itemize}
\item %Rés-C1.1 : Fermeture géométrique.
\end{itemize}
\end{savoir}

\setlength{\parskip}{0ex plus 0.2ex minus 0ex}
 \renewcommand{\contentsname}{}
 \renewcommand{\baselinestretch}{1}



\textit{Ce document est en évolution permanente. Merci de signaler toutes erreurs ou coquilles.}
\tableofcontents

 \renewcommand{\baselinestretch}{1.2}
\setlength{\parskip}{2ex plus 0.5ex minus 0.2ex}



\section{Avant propos}

\subsection{Notion de solide indéformable}
\subsection{Notion de point appartenant à un solide}

\section{Trajectoire d'un point appartenant à un solide}

\section{Vitesse d'un point appartenant à un solide}



\subsection{Définition du vecteur vitesse}

\begin{defi}
\textbf{Vitesse d'un point appartenant à un solide}

Soit un solide $S_0$ auquel on associe le repère $\mathcal{R}_0$ $\left(O_0,\vect{i_0};\vect{j_0},\vect{k_0} \right)$.  Soit un solide $S_1$ auquel on associe le repère $\mathcal{R}_1$,  $\left(O_1,\vect{i_1};\vect{j_1};\vect{k_1} \right)$. Le solide $S_1$ est en mouvement par rapport au solide $S_0$. 

Soit un point $P$ appartenant au solide $S_1$. La vitesse du point $P$ appartenant au solide $S_1$ par rapport au solide $S_0$ se calcule donc ainsi : 
$$
\vect{V(P\in S_1/S_0)}(t) = \left[\dfrac{d\vect{OP(t)}}{dt}\right]_{\mathcal{R}_0}
$$
\end{defi}

\begin{warn}
\begin{minipage}[c]{.15\linewidth}
\begin{center}
\includegraphics[width=.8\textwidth]{png/warning3}
\end{center}
\end{minipage} \hfill
\begin{minipage}[c]{.8\linewidth}
\begin{itemize}
\item Attention à respecter rigoureusement la notation.
\item La vitesse dépend du point d'application.
\end{itemize}
\end{minipage}
\end{warn}


\begin{rem}
\begin{minipage}[c]{.65\linewidth}
Lorsqu'un point est confondu pour deux solides (centre d'une liaison pivot ou d'une liaison rotule par exemple) les vitesses sont égales ainsi, ici : 
$$
\vect{V(0_2\in S_1/S_0)}(t) = \vect{V(0_2\in S_2/S_0)}(t)
$$
\end{minipage}\hfill
\begin{minipage}[c]{.3\linewidth}
\begin{center}
%\includegraphics[width=.8\textwidth]{png/2solides}
\end{center}
\end{minipage}
\end{rem}

\begin{rem}
Dérivée d'un vecteur
\end{rem}

\begin{exemple}
\end{exemple}

\subsection{Vecteur instantané de rotation}
Soit un avion $S_1$ repéré par le repère $\mathcal{R}_1\left(O_1,\vect{i_1},\vect{j_1},\vect{k_1} \right)$ en mouvement libre par rapport à un repère $\mathcal{R}_0\left(O_0,\vect{i_0},\vect{j_0},\vect{k_0} \right)$.  La position de l'avion dans l'espace est repéré par le vecteur $\vect{O_0O_1}=x(t)\vect{i_0}+y(t)\vect{j_0}+z(t)\vect{j_0}$ ainsi que par les angles d'Euler.

\begin{minipage}[c]{.65\linewidth}
\begin{center}
\includegraphics[width=.95\textwidth]{png/euler}
\end{center}
\end{minipage}\hfill
\begin{minipage}[c]{.3\linewidth}
\end{minipage}


Calculons la vitesse du point $O_1$ par rapport à $\mathcal{R}_0$ :
\begin{eqnarray*}
\vectv{O_1}{S_1}{S_0} &=&  \left[\dfrac{d\vect{O_0 O_1}(t)}{dt}\right]_{\mathcal{R}_0} \\
&=& \left[\dfrac{d\left(x(t)\vect{i_0}+y(t)\vect{j_0}+z(t)\vect{j_0} \right)}{dt}\right]_{\mathcal{R}_0}
= 
\left[\dfrac{d\left(x(t)\vect{i_0}\right)}{dt}\right]_{\mathcal{R}_0}
+\left[\dfrac{d\left(y(t)\vect{j_0}\right)}{dt}\right]_{\mathcal{R}_0}
+\left[\dfrac{d\left(z(t)\vect{k_0}\right)}{dt}\right]_{\mathcal{R}_0} \\
&=&
x(t)\underbrace{\left[\dfrac{d\vect{i_0}}{dt}\right]_{\mathcal{R}_0}}_{\vect{0}}
+ \left[\dfrac{dx(t)}{dt}\right]_{\mathcal{R}_0} \vect{i_0}
+y(t)\underbrace{\left[\dfrac{d\vect{j_0}}{dt}\right]_{\mathcal{R}_0}}_{\vect{0}}
+ \left[\dfrac{dy(t)}{dt}\right]_{\mathcal{R}_0} \vect{j_0}
+z(t)\underbrace{\left[\dfrac{d\vect{k_0}}{dt}\right]_{\mathcal{R}_0}}_{\vect{0}} 
+ \left[\dfrac{dz(t)}{dt}\right]_{\mathcal{R}_0} \vect{k_0} \\
&= &\dot{x(t)}\vect{i_0}+\dot{y(t)}\vect{j_0}+\dot{z(t)}\vect{k_0}
\end{eqnarray*}


Soit $P$ un point appartenant à l'avion tel que $\vect{O_1P}=a\vect{i_1}+b\vect{j_1}+c\vect{k_1}$. Calculons la vitesse du point $P$ par rapport à $\mathcal{R}_0$ :

$$
\vectv{P}{S_1}{S_0}=\left[\dfrac{d\vect{O_0 P}(t)}{dt}\right]_{\mathcal{R}_0}
= \left[\dfrac{d\left( \vect{O_0 O_1} + \vect{O_1P}\right)(t)}{dt}\right]_{\mathcal{R}_0}
=\underbrace{\left[\dfrac{d\vect{O_0 O_1}(t)}{dt}\right]_{\mathcal{R}_0}}_{\vectv{O_1}{S_1}{S_0}}
+\left[\dfrac{d\vect{O_1 P}(t)}{dt}\right]_{\mathcal{R}_0}
$$

Calculons donc $\left[\dfrac{d\vect{O_1 P}(t)}{dt}\right]_{\mathcal{R}_0}$ :
\begin{eqnarray*}
\left[\dfrac{d\vect{O_1 P}(t)}{dt}\right]_{\mathcal{R}_0} & = & 
\left[\dfrac{d\left(a\vect{i_1}+b\vect{j_1}+c\vect{j_1} \right)}{dt}\right]_{\mathcal{R}_0} \\
& = & a\left[\dfrac{d\vect{i_1}}{dt}\right]_{\mathcal{R}_0}
+ \underbrace{\left[\dfrac{da}{dt}\right]_{\mathcal{R}_0}}_{\vect{0}} \vect{i_1}
+b\left[\dfrac{d\vect{j_1}}{dt}\right]_{\mathcal{R}_0}
+ \underbrace{\left[\dfrac{db}{dt}\right]_{\mathcal{R}_0}}_{\vect{0}} \vect{j_1}
+c\left[\dfrac{d\vect{k_1}}{dt}\right]_{\mathcal{R}_0} 
+ \underbrace{\left[\dfrac{dc}{dt}\right]_{\mathcal{R}_0}}_{\vect{0}} \vect{k_1} \\
& = & a\left[\dfrac{d\vect{i_1}}{dt}\right]_{\mathcal{R}_0}
+b\left[\dfrac{d\vect{j_1}}{dt}\right]_{\mathcal{R}_0}
+c\left[\dfrac{d\vect{k_1}}{dt}\right]_{\mathcal{R}_0} \\
\end{eqnarray*}

Pour dériver les vecteurs $\vect{i_1}$, $\vect{j_1}$ et $\vect{k_1}$ dans la base $\mathcal{R}_0$ il faut les exprimer dans $\mathcal{R}_0$. On a donc :

\begin{eqnarray*}
\vect{i_1} & = &  \cos\varphi(t) \vect{u} +  \sin\varphi(t) \vect{w}  \\
 & = & \cos\varphi(t)\left(\cos \psi(t) \vect{i_0}+\sin \psi(t) \vect{j_0} \right)+  \sin\varphi(t) \left( \cos \theta(t) \vect{v}+\sin \theta(t) \vect{k_0} \right) \\
 & = & \cos\varphi(t)\left(\cos \psi(t) \vect{i_0}+\sin \psi(t) \vect{j_0} \right)+  \sin\varphi(t) \left( \cos \theta(t) \left(\cos \psi(t) \vect{j_0}-\sin \psi(t) \vect{i_0}  \right)+\sin \theta(t) \vect{k_0} \right) \\
& = & \left(\cos\varphi(t)\cos \psi(t) - \sin\varphi(t) \cos \theta(t)\sin \psi(t) \right)\vect{i_0}  + \left(\cos\varphi(t) \sin \psi(t) + \sin\varphi(t) \cos \theta(t) \cos \psi(t) \right)\vect{j_0} + \sin\varphi(t) \sin \theta(t) \vect{k_0}\\
\vect{j_1} & = & \\
\vect{k_1} & = & \\
\end{eqnarray*}

\subsection{Champ des vecteurs vitesses des points d'un solide}




\section{Accélération d'un point appartenant à un solide}


\begin{thebibliography}{2}
%\bibitem[1]{cite1} Trainer Solo Sport, \textit{Avio et Tiger}, \url{http://www.net-loisirs.com/trainer-solo-sport-p1155.html}.
%\bibitem[2]{cite2} Université Bretagne Sud, \textit{Moteur de modélisme} \url{http://foad.univ-ubs.fr/file.php/1355/TP_meca3d/Moteur_modelisme.zip}.
%\bibitem[3]{JPP} Jean-Pierre Pupier -- Paramétrage -- PTSI -- Lycée Rouvière Toulon.
\end{thebibliography}
\end{document}

