\documentclass[11pt,oneside]{article}
\input{coursHeadings}

\usepackage[%
    pdftitle={Cinématique - Cinématique du solide indéformable - Mécanismes},
    pdfauthor={Xavier Pessoles},
    colorlinks=true,
    linkcolor=blue,
    citecolor=magenta]{hyperref}
\usepackage{schemabloc}



% \makeatletter \let\ps@plain\ps@empty \makeatother
%% DEBUT DU DOCUMENT
%% =================
\sloppy
\hyphenpenalty 10000

\newcommand{\Pointilles}[1][3]{%
\multido{}{#1}{\makebox[\linewidth]{\dotfill}\\[\parskip]
}}


\colorlet{shadecolor}{orange!15}

\newtheorem{theorem}{Theorem}


\begin{document}


\newboolean{prof}
\setboolean{prof}{true}
%------------- En tetes et Pieds de Pages ------------
\pagestyle{fancy}
\renewcommand{\headrulewidth}{0pt}

\fancyhead{}
\fancyhead[L]{%
\noindent\noindent\begin{minipage}[c]{2.6cm}
%Lycée Rouvière PTSI
\includegraphics[width=2cm]{png/logo_ptsi.png}%
\end{minipage}
}

\fancyhead[C]{\rule{12cm}{.5pt}}

\fancyhead[R]{%
\begin{minipage}[c]{3cm}
\begin{flushright}
\footnotesize{\textit{\textsf{Sciences Industrielles\\ pour l'Ingénieur}}}%
\end{flushright}
\end{minipage}
}

\renewcommand{\footrulewidth}{0.2pt}

\fancyfoot[C]{\footnotesize{\bfseries \thepage}}
\fancyfoot[L]{\footnotesize{2012 -- 2013} \\ X. \textsc{Pessoles}}
\ifthenelse{\boolean{prof}}{%
\fancyfoot[R]{\footnotesize{Cours -- CI 2 : Cinématique -- P}}
}{%
\fancyfoot[R]{\footnotesize{Cours -- CI 2 : Cinématique}}
}



\begin{center}
 \huge\textsc{CI 2 -- Cinématique : Modélisation, prévision et vérification du comportement cinématiques des systèmes}
\end{center}

\begin{center}
 \LARGE\textsc{Chapitre 4 -- Cinématique du point dans un mécanisme en mouvement} 
\end{center}

\vspace{.5cm}

\begin{center}
\begin{tabular}{ccc}
\includegraphics[height=3.5cm]{png/fig1} &
\includegraphics[height=3.5cm]{png/fig2}\\
\textit{Modèle CAO d'un} & \textit{Modélisation par}\\
 \textit{arbre à came} & \textit{schéma cinématique}\\
\end{tabular}
\end{center}

\vspace{.2cm}
Dans de nombreux mécanismes, la liaison entre deux solides est modélisée par un contact ponctuel. Cependant, l'écriture du torseur cinématique correspondant au mouvement entre les deux solides n'est pas toujours évidente. 

Intéressons nous par exemple au cas d'un système de distribution présent sur les véhicules équipés de moteurs thermiques. Ce système permet l'admission du mélange air+carburant dans la chambre de combustion et l'échappement des gaz brulés par l'intermédiaire de \textbf{soupapes}. Ces soupapes ont un mouvement de translation. L'ouverture et la fermeture des soupapes est réglée par l'intermédiaire de \textbf{cames} et d'un \textbf{arbre à cames}. La rotation de l'arbre à came est effectuée grâce à un entraînement par une courroie directement reliée au \textbf{vilebrequin} du moteur. 


\begin{prob}
\textsc{Problématique :}
\begin{itemize}
\item Comment calculer les éléments du torseur cinématique dans une liaison de type sphère -- plan.
\end{itemize}
\end{prob}

\begin{savoir}
\textsc{Savoirs :}
\begin{itemize}
\item Connaître les principes de la cinématique des contacts (adhérence ou glissement, roulement et pivotement)
\end{itemize}
\end{savoir}

\setlength{\parskip}{0ex plus 0.2ex minus 0ex}
 \renewcommand{\contentsname}{}
 \renewcommand{\baselinestretch}{1}

\tableofcontents

 \renewcommand{\baselinestretch}{1.2}
\setlength{\parskip}{2ex plus 0.5ex minus 0.2ex}

% \vspace{1cm}
\textit{Ce document est en évolution permanente. Merci de signaler toutes
erreurs ou coquilles.}


\section{Présentation -- Système de distribution}

\begin{minipage}[c]{.65\linewidth}
Intéressons nous à la modélisation d'une soupape, notée $S_1$ en liaison glissière d'axe $(A,\vect{y_0})$ avec le moteur $S_0$. La came $S_2$ est en liaison pivot d'axe $(O,\vect{z_0})$ avec $S_0$.  $S_1$ et $S_2$ sont en contact ponctuel de normale $(I,\vect{y_0})$.
\end{minipage}
\begin{minipage}[c]{.3\linewidth}
\begin{center}
\includegraphics[width=.8\textwidth]{png/fig2} 
\end{center}
\end{minipage}
L'objectif de ce cours est de calculer en $I$ le torseur cinématique suivant :


\ifthenelse{\boolean{prof}}{%
\begin{center}
  \begin{tabular}{|>{\columncolor[gray]{.95}}p{0.6\textwidth}|}
\hline
$$
\{\mathcal{V}(S_2/S_1)\}
=\left\{
\begin{array}{c}
\vect{\Omega(S_2/S_1)}\\
\vect{V(I,S_2/S_1)}
\end{array}
\right\}_I
$$ \\
\hline
\end{tabular}
\end{center}
}{
\begin{center}
\begin{tabular}{|p{8cm}|}
\hline
\\
\\
\\
\\
\hline
\end{tabular}
\end{center}
}


\section{Modélisation des vitesses de glissement}
\subsection{Hypothèses}
Considérons deux solides $S_1$ et $S_2$ en contact ponctuel. 

Définissons alors $I$ le point de contact entre les deux solides et $\vect{n_{12}}$ la normale de contact. On appelle $\mathcal{P}$ le plan normal à $\vect{n_{12}}$ en $I$. Il est tangent à $S_1$ et $S_2$. 

On note $
\{\mathcal{V}(S_2/S_1)\}
=\left\{
\begin{array}{c}
\vect{\Omega(S_2/S_1)}\\
V(I,S_2/S_1)
\end{array}
\right\}_I$
le torseur cinématique du mouvement relatif de $S_2$ par rapport à $S_1$ au point $I$.


\begin{center}
\includegraphics[width=.8\textwidth]{png/ContactReel} 
\end{center}

\subsection{Vitesses de rotation}
On a vu que dans le cas d'un contact ponctuel il existait 3 degrés de libertés de rotation paramétrables par les angles d'Euler. Nous ne cherchons pas ici à calculer directement le vecteur $\vect{\Omega(S_2/S_1)}$ en fonction de ces angles.

Cependant, ce vecteur est décomposable en une somme de deux vecteurs.
\subsubsection*{Le vecteur de pivotement}
Ce vecteur est normal au plan $\mathcal{P}$. On le note $\vect{\Omega_p(S_2/S_1)}$.

\subsubsection*{Le vecteur de roulement}
Ce vecteur est contenu dans le plan $\mathcal{P}$. On le note $\vect{\Omega_r(S_2/S_1)}$.


\begin{resultat}
\textbf{Vitesse de rotation}

La vitesse de rotation se compose donc ainsi :
$$\vect{\Omega(S_2/S_1)} = \vect{\Omega_p(S_2/S_1)} + \vect{\Omega_r(S_2/S_1)}$$
\end{resultat}



\subsection{Vitesse de glissement}
\subsubsection{Position du point de contact entre solides}
Cinématiquement, le point $I$ n'est pas unique. En effet, on peut distinguer l'existence de 3 points différents :
\begin{itemize}
\item le point I matériel appartenant au solide $S_1$;
\item le point I matériel appartenant au solide $S_2$;
\item le point I (non matériel) correspondant au point de contact entre les deux solides.
\end{itemize}

À l'instant $t$, ces points peuvent être confondus. À $t+dt$ ils peuvent être distincts.

En conséquence :
\begin{resultat}
$$
\vect{V(I,S_2/S_0)} \neq \vect{V(I,S_1/S_0)}
$$ 
\end{resultat}

\begin{exemple}
\begin{center}
\includegraphics[width=.4\textwidth]{png/ptI}
\end{center}
\end{exemple}

\subsubsection{Définition de la vitesse de glissement}
On appelle vecteur vitesse de glissement en $I$ de $S_2/S_1$ le vecteur $\vect{V(I,S_2/S_1)}$.

\begin{rem}
On considère qu'il y a toujours contact entre $S_1$ et $S_2$ et que les solides sont indéformables. En conséquence :
$$\vect{V(I,S_2/S_1)} \in \mathcal{P}$$
\end{rem}

\subsubsection{Condition de roulement sans glissement}
\begin{defi}
\textbf{Condition de roulement sans glissement}

Dans de très nombreux mécanismes (dans les engrenages, lors du contact entre la roue et le sol \textit{etc}) on peut faire l'hypothèse que le glissement est nul. On a alors : 

$$
\vect{V(I,S_2/S_1)} = \vect{0}
$$
\end{defi}


Il est alors possible d'identifier des lois de comportement.
\subsection{Méthode de calcul de la vitesse de glissement}
Le calcul de la vitesse de glissement peut se calculer à l'aide de la procédure suivante.

\begin{methode}
\begin{enumerate}
\item Paramétrer le système
\item Décomposer le mouvement : $\vect{V(I,S_1/S_2)}=\vect{V(I,S_1/S_0)}+\vect{V(I,S_0/S_2)}$
\item Calculer $\torseurcin{V}{S_1}{S_0}$ au point $I$
\item Calculer $\torseurcin{V}{S_2}{S_0}$ au point $I$
\item Calculer  $\vect{V(I,S_1/S_2)}$
\end{enumerate}
\end{methode}

\begin{warn}
\textbf{$I$ n'est pas un point matériel. Il n'appartient ni à $S_1$ ni à $S_2$. On ne peut donc pas calculer $\left[\dfrac{\vect{OI}}{dt}\right]_{\mathcal{R}_0}$}.
\end{warn}

\newpage

\section{Application -- Calcul de la vitesse de glissement entre la came et la soupape}
Pour le système de distribution composé d'une came, d'une soupape et du moteur, on se propose de calculer la vitesse de glissement entre la soupape et la came.
\subsection*{Paramétrage}
\begin{center}
\includegraphics[width=.7\textwidth]{png/parametrage} 
\end{center}

\subsection*{Décomposition en mouvements simples}
D'après la composition des mouvements, on a : 
\begin{center}
\begin{tabular}{|p{8cm}|}
\hline
$$
\vect{V(I,S_2/S_1)} = \vect{V(I,S_1/S_0)} + \vect{V(I,S_0/S_2)}
$$
\\
\hline
\end{tabular}
\end{center}

\subsection*{Calcul de $\{\mathcal{V}(S_1/S_0)\}$}

Nature du mouvement entre $S_1$ et $S_0$ : liaison glissière d'axe $(A,\vect{y_0})$.
On a donc : 
$$
\{\mathcal{V}(S_1/S_0)\} =
\left\{
\begin{array}{c}
\vect{\Omega(S_1/S_0)} = \vect{0} \\
\vect{V(A,S_1/S_0)} = \dfrac{d\vect{OA}}{dt} \\
\end{array}
\right\}_A
$$

Calculons $\vect{V(A,S_1/S_0)}$ :
$$
\vect{V(A,S_1/S_0)} = \dfrac{d\vect{OA}}{dt} = \dfrac{d\lambda(t)\vect{y_0}}{dt} =  e \dot{\theta}(t) \cos \theta(t) \vect{y_0}
$$

Calculons alors $\vect{V(I,S_1/S_0)}$ :
$$
\vect{V(I,S_1/S_0)} = \vect{V(A,S_1/S_0)} + \underbrace{\vect{IA} \wedge \vect{\Omega(S_1/S_0)}}_{\vect{0}}
$$

D'où :
\begin{center}
\begin{tabular}{|p{8cm}|}
\hline
$$
\{\mathcal{V}(S_1/S_0)\} =
\left\{
\begin{array}{c}
\vect{\Omega(S_1/S_0)} = \vect{0} \\
\vect{V(I,S_1/S_0)} = e \dot{\theta}(t) \cos \theta(t) \vect{y_0}  \\
\end{array}
\right\}_I
$$
\\
\hline
\end{tabular}
\end{center}

\subsection*{Calcul de $\{\mathcal{V}(S_2/S_0)\}$}
Nature du mouvement entre $S_2$ et $S_0$ : liaison pivot d'axe $(O,\vect{z_0})$.
On a donc : 

$$
\{\mathcal{V}(S_2/S_0)\} =
\left\{
\begin{array}{c}
\vect{\Omega(S_2/S_0)} = \dot{\theta}\vect{z_0} \\
\vect{V(A,S_2/S_0)} = \vect{0} \\
\end{array}
\right\}_O
$$

Calculons $\vect{V(I,S_2/S_0)}$ :
$$
\vect{V(I,S_2/S_0)} = 
\underbrace{\vect{V(O,S_2/S_0)}}_{\vect{0} }
+ \vect{IO} \wedge \vect{\Omega(S_2/S_0)} = \left(R\vect{y_0}-e\vect{x_2}\right) \wedge \dot{\theta}\vect{z_0}
$$

$$
\vect{V(I,S_2/S_0)} = R\dot{\theta}\vect{x_0} + e\dot{\theta}\vect{y_2}
$$

D'où :
\begin{center}
\begin{tabular}{|p{8cm}|}
\hline
$$
\{\mathcal{V}(S_2/S_0)\} =
\left\{
\begin{array}{c}
\vect{\Omega(S_2/S_0)} = \dot{\theta}\vect{z_0} \\
\vect{V(I,S_2/S_0)} = R\dot{\theta}\vect{x_0} + e\dot{\theta}\vect{y_2}  \\
\end{array}
\right\}_I
$$ \\
\hline
\end{tabular}
\end{center}
\subsection*{Calcul de $\vect{V(I,S_2/S_1)}$}
On a :
$$
\vect{V(I,S_2/S_1)} = \vect{V(I,S_1/S_0)} + \vect{V(I,S_0/S_2)}
$$

$$
\Longleftrightarrow \vect{V(I,S_2/S_1)} = \vect{V(I,S_1/S_0)} - \vect{V(I,S_2/S_0)}
$$

$$
\Longleftrightarrow \vect{V(I,S_2/S_1)} =  e \dot{\theta}(t) \cos \theta(t) \vect{y_0} - \left( R\dot{\theta}\vect{x_0} + e\dot{\theta}\vect{y_2} \right)
$$

$$
\Longleftrightarrow \vect{V(I,S_2/S_1)} = 
 e \dot{\theta}(t) \cos \theta(t) \vect{y_0} 
- R\dot{\theta}\vect{x_0} 
- e\dot{\theta}\left( \cos \theta(t) \vect{y_0}- \sin \theta(t) \vect{x_0} \right)
$$

Au final :
\begin{center}
\begin{tabular}{|p{8cm}|}
\hline
$$
\vect{V(I,S_2/S_1)} =  \dot{\theta}\left(e\sin \theta(t) - R\right)\vect{x_0} 
$$\\
\hline
\end{tabular}
\end{center}
\end{document}