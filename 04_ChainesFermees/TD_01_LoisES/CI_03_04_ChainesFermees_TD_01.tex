\documentclass[10pt]{article}
\input{style/coursHeadings}
\input{style/programHeadings}
\input{style/macros_SII}
\input{style/macros_Titres}
\input{style/macros_Frames}

%Si le boolen xp est vrai : compilation pour xabi
%Sinon compilation Damien
\newboolean{xp}
\setboolean{xp}{true}

\newboolean{prof}
\setboolean{prof}{true}

\newboolean{td}
\setboolean{td}{true}

\usepackage[%
    pdftitle={},
    pdfauthor={Xavier Pessoles},
    colorlinks=true,
    linkcolor=blue,
    citecolor=magenta]{hyperref}

\def\discipline{Sciences Industrielles de l'Ingénieur}

\def\xxtitre{\ifthenelse{\boolean{xp}}{CI 3 -- CIN : Étude du comportement cinématique des systèmes}{}}

\def\xxsoustitre{\ifthenelse{\boolean{xp}}{
Chapitre 4 -- Étude des chaînes fermées : Détermination des lois Entrée -- Sortie}{
}}


\def\xxauteur{\ifthenelse{\boolean{xp}}{
\noindent 2013 -- 2014 \\
Xavier \textsc{Pessoles}}{
}}


\def\xxpied{\ifthenelse{\boolean{xp}}{
CI 4 : Cinématique \\
Ch 4 : Chaînes fermées -- Lois entrée -- sortie -- TD -- \ifthenelse{\boolean{prof}}{P}{E}%
}{
}}

\def\xxcathegorie{\ifthenelse{\boolean{xp}}{
2013 -- 2014 \\
Xavier \textsc{Pessoles}}{
Informatique - Cours}}





%---------------------------------------------------------------------------


\begin{document}

\ifthenelse{\boolean{xp}}{\input{style/enteteXP}}{\input{style/enteteDI}}



%\renewcommand{\baselinestretch}{1.2}
%\setlength{\parskip}{2ex plus 0.5ex minus 0.2ex}



\section*{Prothèse active transtibiale}

\begin{flushright}
\textit{D'après concours Mines-Ponts -- MP -- 2013.}
\end{flushright}
\begin{minipage}[c]{.8\linewidth}
Les ingénieurs du MIT ont mis au point une prothèse active permettant aux personnes amputées en dessous du genou d'avoir une marche s'approchant d'une marche d'une personne valide. 
\begin{obj} 
Dans le but de dimensionner le vérin à utiliser sur la prothèse, on cherche à dimensionner sa course utile. D'autre part, la connaissance du modèle mécanique de transmission est nécessaire afin de renseigner un modèle multiphysique. 
\end{obj}

 On donne un extrait du cahier des charges.
 
\end{minipage} \hfill
\begin{minipage}[c]{.15\linewidth}
\begin{center}
\includegraphics[width=\textwidth]{images/prot_01}
\end{center}
\end{minipage}

\begin{minipage}[c]{.33\linewidth}
\begin{center}
\includegraphics[width=\textwidth]{images/uc}

\textit{Diagramme de cas des utilisations}
\end{center}
\end{minipage} \hfill
\begin{minipage}[c]{.63\linewidth}
\begin{center}
\includegraphics[width=\textwidth]{images/exigences}

\textit{Diagramme d'exigences}
\end{center}
\end{minipage}

\vspace{.25cm}

La structure interne du système est donnée par les figures ci-contre. Le paramétrage géométrique est donné ci-dessous.

\vspace{.25cm}


\begin{minipage}[c]{.3\linewidth}
\begin{center}
\includegraphics[width=\textwidth]{images/prot_02}

\textit{Représentation volumique}
\end{center}
\end{minipage}
 \hfill
\begin{minipage}[c]{.65\linewidth}
\begin{center}
\includegraphics[width=\textwidth]{images/Systeme}

\textit{Diagramme de blocs internes}
\end{center}
\end{minipage}


\begin{minipage}[c]{.54\linewidth}

Le repère $\rep{0} (O, \vx{},\vy{0} ,\vz{0})$ est lié au tibia noté 0 fixe dans toutes nos études. Ce repère est supposé galiléen (hypothèse justifiée dans le sujet).

Le repère $\rep{1} (O, \vx{},\vy{1} ,\vz{1} )$ est lié au pied artificiel noté 1, supposé indéformable. On note $\theta (t)=(\vy{0} ,\vy{1})=(\vz{0},\vz{1})$ l'angle de rotation du pied par rapport au tibia. D'autre part, le vecteur unitaire $\vect{n_1}$ définit la direction des ressorts avec $\delta=(\vy{1} ,\vect{n_1})$ considéré comme constant tout au long
du cycle de marche.

Le repère $\rep{2} (O, \vx{} ,\vy{2} ,\vz{2} )$ est lié au basculeur noté 2. On note $\alpha(t) =(\vy{0} ,\vy{2})=(\vz{0} ,\vz{2})$ l'angle de rotation du basculeur par rapport au tibia.

Le repère $\rep{3} ( A, \vx{},\vy{3} ,\vz{3})$ est lié au vérin électrique 3. On note $\beta(t )=(\vy{0} ,\vy{3})=(\vz{0} ,\vz{3})$ l'angle de rotation du vérin électrique par rapport au tibia. Le vérin électrique comporte une tige notée $3_1$ et un
corps noté $3_2$.

On pose : $\vect{OA}=a\vect{z_0}$, $\vect{BA}=\lambda(t)\vy{3}$, $\vect{BO}=b\vy{2}$, $\vect{BO} = b\vect{z_2}$ avec $b=0,039 \; m$ et $a=0,117\;m$.

En l'absence d'action sur la prothèse, une position repos est identifiée par les paramètres $\theta_R$, $\alpha_R$, et $\delta_R$. Cette position est notamment obtenue lorsque le tibia est vertical et que le pied est en appui horizontalement
sur le sol. Les valeurs numériques sont alors : $\theta_R=0\textdegree$ , $\alpha_R=9\textdegree$ et $\delta_R=\delta=-17\textdegree$.
\end{minipage} \hfill
\begin{minipage}[c]{.43\linewidth}
\begin{center}
\includegraphics[width=\textwidth]{images/prot_03}

\textit{Modélisation cinématique pour $\theta=0\textdegree$}
\end{center}
\end{minipage}

\subparagraph{}
\textit{Après avoir identifié les différents paramètres variables du système, préciser quelle est l'entrée et quelle est la sortie. Quelle relation lie les paramètres $\alpha$ et $\theta$ ?} 

\subparagraph{}
\textit{Paramétrer le système et réaliser les figures planes correspondant aux différents changements de repères.} 

\subparagraph{}
\textit{Déterminer la loi entrée-sortie correspondant à la modélisation proposée.} 
\end{document}


