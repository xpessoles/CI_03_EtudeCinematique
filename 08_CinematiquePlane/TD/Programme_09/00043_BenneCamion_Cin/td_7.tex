\documentclass[11pt,oneside]{article}
\input{coursHeadings}

\usepackage[%
    pdftitle={TD Cinématique},
    pdfauthor={Xavier Pessoles},
    colorlinks=true,
    linkcolor=blue,
    citecolor=magenta]{hyperref}



% \makeatletter \let\ps@plain\ps@empty \makeatother
%% DEBUT DU DOCUMENT
%% =================
\sloppy
\hyphenpenalty 10000

\newcommand{\Pointilles}[1][3]{%
\multido{}{#1}{\makebox[\linewidth]{\dotfill}\\[\parskip]
}}


\begin{document}


\newboolean{prof}
\setboolean{prof}{true}
%------------- En tetes et Pieds de Pages ------------
\pagestyle{fancy}
\renewcommand{\headrulewidth}{0pt}

\fancyhead{}
\fancyhead[L]{%
\begin{minipage}[c]{1.6cm}
\includegraphics[width=1.4cm]{png/logo_ptsi.png}%
\end{minipage}
\rule{2cm}{.5pt}
}

\fancyhead[C]{\rule{12cm}{.5pt}}

\fancyhead[R]{%
\begin{minipage}[c]{3cm}
\begin{flushright}
\footnotesize{\textit{\textsf{Sciences Industrielles\\ pour l'Ingénieur}}}%
\end{flushright}
\end{minipage}
}

\renewcommand{\footrulewidth}{0.2pt}

\fancyfoot[C]{\footnotesize{\bfseries \thepage}}
%\fancyfoot[L]{\footnotesize{2011 -- 2012} \\ X. \textsc{Pessoles}}
\ifthenelse{\boolean{prof}}{%
%\fancyfoot[R]{\footnotesize{TD -- CI 2 : Cinématique -- P}}
}{%
%\fancyfoot[R]{\footnotesize{TD -- CI 2 : Cinématique}}
}


%\begin{center}
%\textit{Centre d'intérêt}
%\end{center}

\begin{center}
 \LARGE\textsc{CI 2 -- Cinématique : Modélisation, prévision et vérification du comportement cinématiques des systèmes}
\end{center}

\begin{center}
 \Large\textsc{Chapitre 5 -- Étude graphique des mouvements plans} 
\end{center}

\vspace{1cm}

\section{Benne de camion}

\setcounter{paragraph}{0}
On se propose d'étudier le système qui assure l'ouverture d'une benne de camion de ramassage d'ordures.


\begin{center}
\includegraphics[width=.9\textwidth]{png/fig3}
\end{center}

L'objectif est de vérifier le critère de la FS1. Le schéma cinématique de la mise en mouvement du système est fourni sur la figure suivante. Un vérin impose le mouvement du système. Dans la position donnée, la vitesse de sortie de la tige 2 par rapport au corps du vérin 1 est de $0,1\; m/s$ (Echelle des vitesses : 3cm pour 0,1 m/s).


\begin{center}
\includegraphics[width=.9\textwidth]{png/fig4}
\end{center}

\paragraph{}
\textit{Déterminer graphiquement avec les justifications utiles $\vectv{B}{3}{0}$ puis $\vectv{F}{3}{0}$}


\paragraph{}
\textit{Déterminer $\omega(3/0)$ et conclure vis-à-vis du cahier des charges $(BO=6m)$.}

La benne est munie d'une porte 4 qui s'ouvre lorsque 3 s'incline.


\paragraph{}
\textit{Déterminer graphiquement avec les justifications utiles $\vectv{C}{5}{0}$ et $\vectv{G}{5}{0}$.}


\paragraph{}
\textit{Déterminer $\omega_{5/3}$.}

Constructions graphiques pour déterminer $\vectv{B}{3}{0}$.
\begin{center}
\includegraphics[width=.7\textwidth]{png/fig5}
\end{center}


Constructions graphiques pour déterminer $\vectv{F}{3}{0}$.
\begin{center}
\includegraphics[width=.7\textwidth]{png/fig5}
\end{center}


Constructions graphiques pour déterminer $\vectv{C}{5}{0}$.
\begin{center}
\includegraphics[width=.7\textwidth]{png/fig5}
\end{center}

Constructions graphiques pour déterminer $\vectv{G}{5}{0}$.
\begin{center}
\includegraphics[width=.7\textwidth]{png/fig5}
\end{center}

\newpage


 
\end{document}