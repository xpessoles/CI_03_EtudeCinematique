\documentclass[11pt,oneside]{article}
\input{coursHeadings}

\usepackage[%
    pdftitle={TD Cinématique},
    pdfauthor={Xavier Pessoles},
    colorlinks=true,
    linkcolor=blue,
    citecolor=magenta]{hyperref}



% \makeatletter \let\ps@plain\ps@empty \makeatother
%% DEBUT DU DOCUMENT
%% =================
\sloppy
\hyphenpenalty 10000

\newcommand{\Pointilles}[1][3]{%
\multido{}{#1}{\makebox[\linewidth]{\dotfill}\\[\parskip]
}}


\begin{document}


\newboolean{prof}
\setboolean{prof}{true}
%------------- En tetes et Pieds de Pages ------------
\pagestyle{fancy}
\renewcommand{\headrulewidth}{0pt}

\fancyhead{}
\fancyhead[L]{%
\begin{minipage}[c]{1.6cm}
\includegraphics[width=1.4cm]{png/logo_ptsi.png}%
\end{minipage}
\rule{2cm}{.5pt}
}

\fancyhead[C]{\rule{12cm}{.5pt}}

\fancyhead[R]{%
\begin{minipage}[c]{3cm}
\begin{flushright}
\footnotesize{\textit{\textsf{Sciences Industrielles\\ pour l'Ingénieur}}}%
\end{flushright}
\end{minipage}
}

\renewcommand{\footrulewidth}{0.2pt}

\fancyfoot[C]{\footnotesize{\bfseries \thepage}}
%\fancyfoot[L]{\footnotesize{2011 -- 2012} \\ X. \textsc{Pessoles}}
\ifthenelse{\boolean{prof}}{%
%\fancyfoot[R]{\footnotesize{TD -- CI 2 : Cinématique -- P}}
}{%
%\fancyfoot[R]{\footnotesize{TD -- CI 2 : Cinématique}}
}


%\begin{center}
%\textit{Centre d'intérêt}
%\end{center}

\begin{center}
 \LARGE\textsc{CI 2 -- Cinématique : Modélisation, prévision et vérification du comportement cinématiques des systèmes}
\end{center}

\begin{center}
 \Large\textsc{Chapitre 5 -- Étude graphique des mouvements plans} 
\end{center}

\vspace{1cm}


\section{Porte d'autobus}
\setcounter{paragraph}{0}
On considère un système d'ouverture de porte d'autobus dont on donne un extrait de cahier des charges ci-dessous.

\begin{center}
\includegraphics[width=.9\textwidth]{png/fig6}
\end{center}
 
La figure de la page suivante représente le schéma du mécanisme actionneur d'une porte (3) d'autobus (en vue dessus). Au dessus de la porte, un vérin pneumatique (air comprimé) (4,5) entraîne une bielle (2) en liaison pivot avec la carrosserie (1). Le bras (AB), encastré à la bielle (2), entraîne le battant de porte (3) qui est guidé par un maneton (C) se déplaçant dans la rainure. L'amplitude de rotation de la bielle (2) de 90 degrés environ permet d'obtenir les positions extrêmes (ouvert/fermé) du battant (3). 

Pour tous les tracés des vitesses on prendra 10mm/s pour 5mm.
 La vitesse de sortie du vérin lors de l'ouverture de la porte d'autobus est $||\vectv{F}{4}{5}||=50mm/s$

\paragraph{}
\textit{Déterminer graphiquement le vecteur vitesse $\vectv{F}{4}{1}$ en justifiant la démarche suivie. }

\paragraph{}
\textit{Déterminer, par équiprojectivité, le vecteur vitesse $\vectv{B}{3}{1}$ en justifiant la démarche suivie.}

\paragraph{}
\textit{Donner la direction du vecteur vitesse $\vectv{C}{3}{1}$. En déduire la position du centre instantané de rotation de la porte (3) par rapport au bâti (1) noté $I_{31}$.}

\paragraph{}
\textit{Déterminer graphiquement le vecteur vitesse $\vectv{C}{3}{1}$ en justifiant la démarche suivie.}

\paragraph{}
\textit{Conclure quant à la capacité de  la porte d'autobus à satisfaire le critère de vitesse de coulissement du maneton C. }

\paragraph{}
\textit{Déterminer le CIR du mouvement de (4) par rapport à 1.}

\newpage

$$\quad$$

\vspace{15cm}

\begin{center}
\includegraphics[width=.8\textwidth]{png/fig7}
\end{center}
 
\end{document}